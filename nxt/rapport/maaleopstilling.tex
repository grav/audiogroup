\section{M�leopstillinger}
Praktiske omst�ndigheder
Hvad grej, hvad software, beskriv placering (12 cm, 10 grader osv)



Mikkel:
P� hardware-siden har vi benyttet en b�rbar Mac tilkoblet et Alesis IO|14 lydkort. 

Vi har foretaget m�lingerne ved hj�lp af softwaren FuzzMeasure 2.

Ved hj�lp af FuzzMeasure har vi sendt et sinus-sweep (1000 ms varighed, g�ende fra 100 Hz til 20000 Hz) igennem NXT-h�jttaleren og optaget det igen. FuzzMeasure beregner impulssvaret ved at affolde sinus-sweepet fra det optagede signal.


Beskriv fuzz
hardware, alesis-lydkort (headphone-amp,preamps?)

Bent:
\begin{figure}[h!]
\begin{center}
\includegraphics[width=12cm]{ecm8000_freqres.pdf}
\end{center}
\caption{Frekvensgangen p� Behringer ECM8000 mikrofonen}
\label{ecm8000}
\end{figure}

Vi har m�lt i 10 graders interval op til 40 grader fra on axis og i alle retninger. I alt 81 m�linger.\\
Det giver anledning til f�lgende grid over m�lingerne:

Vores closerange m�linger er alle lavet ved 12 cm afstand, hhv. med og uden kabinet.\\

Desuden har vi foretaget m�linger p� 2 m. afstand i en vinkel p� ca. 45 grader oppe fra. Der er her lavet m�linger hvor h�jtaleren peger hen imod lytteren, og v�k fra lytteren, for at afspejle hvad man kan kalde almindeligt brug.

Alle m�lingerne er foretaget i et lydd�dt rum, med en udefra d�mpning p� -60DB. Rummet er approksimativt $60m^3$ med en nedre begr�nsende frekvens p� 100Hz, svarende til starten p� vores sweep (100Hz - 20kHz).

\begin{figure}[h!]
\begin{center}
\includegraphics[width=12cm]{room.jpg}
\end{center}
\caption{M�linger blev foretaget i et lydd�dt rum.}
\label{room}
\end{figure}
