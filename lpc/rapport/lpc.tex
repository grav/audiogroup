\section{LPC-analyse og syntese}
Til LPC-analysen har vi benyttet RT_LPC som en blackbox. Måden hvorpå LPC-analysen foregår er beskrevet på http://wiki.cs.princeton.edu/index.php/RT_LPC.

RT_LPC giver mulighed for at detektere pitch vha. autokorrelation. Vi har dog istedet implementeret average magnitude difference (AMDF) pitch detection på baggrund af [Sayood] pp. 543. Formlen for AMDF er:

[formel],

hvor p er perioden. For hver frame analyserer vi med værdier af p mellem 2.5 og 19.5 ms, svarende til 860 Hz ned til 110 Hz, hvor menneskelig tale normalt befinder sig. Herefter vælger vi den minimale AMDF-værdi som vores pitch.

I følgende eksempel analyseres 20 ms af 'y'-lyden i 'tydeligt'. X-aksen betegner p-værdien i samples, og Y-aksen betegner AMDF-værdien.

[graf]

Det ses at minimum ligger ved ca 730 samples, svarende til ... Hz.

Er den minimale værdi over et bestemt threshold, vælger vi at opfatte frame'n som en ustemt lyd.

Ligesom ved analysedelen af LPC, benytter vi også RT_LPC til at syntetisere talen, således at vi blot giver de ved analysen fundne koefficienter samt den beregnede pitch videre. Som beskrevet på http://wiki.cs.princeton.edu/index.php/RT_LPC skifter RT_LPC mellem et pulstog og et støj-signal afhængigt af pitch'en.

