% -*- coding: latin-1 -*-
\section{Konklusion}
I dette projekt har vi konstrueret en offline taleenkoder/dekoder vha. LPC.

Vi har implementeret pitch-detection vha. AMDF, hvilket har givet meget overbevisende resultater, n�r man s�rger for ikke at fors�ge at detektere perioder l�ngere end det data man arbejder med.

Vi har implementeret 50\% overlap for at lave bl�dere overgange mellem de forskellige frames. Dette er ogs� lykkedes, dog med visse sideeffekter i form af introducerede undertoner (som muligvis ogs� afh�nger af vinduest�rrelsen) samt en flanger-lignende effekt. Sidstn�vnte kan skyldes at vi i to p� hindanden-f�lgende frames detekterer to n�rliggende pitches og derefter adderer disse frames. S�ledes f�r den resulterende frame pr�g af to t�tliggende toner, hvilket kan give den karakteristiske flanger-effekt.

Vi har som sagt benyttet LPC-algoritmen som blackbox s� uden at skulle g�re os kloge p� pr�cis hvorfor, mener vi at det giver en v�sentlig kvalitetsforbedring at for�ge antallet af koefficienter fra 12 til 20, mens en for�gelse fra 20 til 32 ikke giver en s� udpr�get forbedring.

Det har v�ret sjaw!

