% -*- coding: utf-8 -*-
\subsection*{Biquad-maskering}
De tidligere beskrevne maskeringsmetoder benytter sig kun af
informationer fra de to omkringliggende bånd. Biquad maskering er et
forsøg på at lave alle bånd kunne maskere hinanden, uden at de ved et
uheld kommer til at maskere sig selv.\\
Maskeringskurven er frembragt på gølgende vis:
\begin{itemize}
\item Generer en blok med hvidstøj.
\item For hver bånd maxværdi som ligger over ath kurven, lav en parametrisk
  equalisering af hvidstøjen, hvor gain er styret af signal-to-ath
  ratio og båndbredden er fikseret til 1/2 oktav. Dette gøres ikke for
  båndet selv, idet det således ville kunne overkygge sig selv.
\item FFT signalet, så vi kommer tilbage i frekvensdomænet
\item Smooth grafen
\item Lav maskeingsopslaget for det givne bånd.
\end{itemize}
Dette gøres dynamisk for hver frame, bortset fra
hvidstøjsgenereringen. Den laves kun een gang og genbruges derefter.
