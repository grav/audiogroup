% -*- coding: utf-8 -*-
\section{Brugervejledning}
Koden kan afvikles direkte på Mac og Linux. Den kan sandsynligvis også
afvikles på andre platforme, men det vil nok kræve lidt tilpadsning af
udviklingsmiljøet.\\
\\
Programmet styres af en række parametre, som står hardcodede i
\texttt{config.cc}. Disse er flg.:
\begin{verbatim}
float config::ath_weight = 1.0;
float config::mask_weight = 1.0;
float config::quality = 1.0;
mask_t config::mask = MASK_MAX;
filter_t config::filter = FILTER_LOG;
unsigned int config::num_threads = 3;
\end{verbatim}
\texttt{ath\_weight} er en vægtning af ATH kurven. Denne skal have
værdier mellem 0 og 1.\\
\texttt{mask\_weight} er en vægtning af maskeringen, denne kan ligeledes
have værdier mellem 0 og 1.\\
\texttt{quality} parameteren styrer bittildelings algoritmen
(quantizeren). Værdierne kan være mellem 0 og 1, hvor 1 er fuld pcm
kvalitet og 0 er stilhed. Denne parameter kan være lidt svær at tyre,
men kan med lidt snilde kontrollere output bitraten.\\
\texttt{mask} og \texttt{filter} styrer begge hvilke algortimer der skal
bruges internt. Der findes flg. maskeringsalgortimer: \texttt{MASK\_MAX},
\texttt{MASK\_BIQUAD} og \texttt{MASK\_AVG}, samt flg. filtrerings
algortimer: \texttt{FILTER\_LIN} og \texttt{FILTER\_LOG}.\\
\texttt{num\_threads} styrer som navnet antyder, hvor mange tråde
programmet skal bruge. Dette kan med fordel sættes til antalles et
tilgængelige cpuer gange to, minus et.\\
\\
Disse parametre kan frit ændres, hvorefter programmet skal
genkompileres. Dette gøre ganske enkelt med kommandoen \texttt{make}.\\
\\
Udførslen af programmet foretages med følgende kommando linie, med
dertilhørende parametre: \texttt{./soundenc inputfil}.\\
Det er vigtigt at inputfilen er i et wavformat (signed/unsigned,
8/16/32 bit, integer/float baseret), samt at filen er i mono.\\
Overholdes dette ikke giver programmet udefinerede resultater.\\
\\
Programmet spiller automatisk det beregnede resultat når beregningerne
er fuldført, samt lagrer resultatet i filen \texttt{output.wav}, som en 32
bit floatingpoint PCM fil.\\
\\
\textbf{Bemærk: output.wav overskrives uden varsel hvis den findes i
  forvejen!}\\
\\
Efter encodningen vises en estimeret filstørrelse samt bitrate. Disse tal er ikke eksakte, men blot estimater beregnet på baggrund af bit-tildelingen (se afsnit \ref{sec.quantization}) samt ideen om subbåndsnedsampling (se afsnit \ref{sec.subband}).\\
\\
Koden er afhængig af flg. open source libraries for at kunne kompilere:\\
libsndfile - \texttt{http://www.mega-nerd.com/libsndfile}\\
libfftw3 - \texttt{http://www.fftw.org}\\
libao - \texttt{http://www.xiph.org/ao}
