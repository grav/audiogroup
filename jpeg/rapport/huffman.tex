Efter kvantisering har vi reduceret en r�kke v�rdier i vores koefficientmatricer til 0'er, og vi kan nu udnytte dette til komprimering ved hj�lp af Huffman-kodning.

Iflg. JPEG-standarden encodes DC-differencensen samt AC-v�rdierne forskelligt. 

Vi har undladt at implementere DC encoding, og i stedet AC encoder vi alle indgange i koefficient-tabellen. Dette giver naturligvis ikke en liges� optimal komprimeringsfaktor.

Frem for at generere en Huffman tabel til et specifikt billede, benytter vi de anbefalede tabeller fra JPEG standard-specifikationen, en for luminans-kanalen, og en for de to krominsanskanaler.

Tabellerne er hardcoded i filen huffman\_tables.h. 

Ved encoding beregnes et (runlength, size) par ud fra en koefficient (samt eventuelle f�rkommende 0-koefficienter). Parrent sl�es op i den relevante tabel, hvorefter resultatet p�h�ftes vores bitstream sammen med en bin�r encoding af koefficienten.

Ved decoding genereres et bin�rt s�getr� ud fra den relevante tabel. Vi fors�ger derp� at sl� en delstreng af bitstreamen op, hvor delstrengen bliver en bit l�ngere, indtil s�getr�et returnerer et (runlength, size) par, hvorefter vi afkoder en koefficient (samt eventuelle 0-koefficienter) og gentager processen.

Vi har implementeret decoding men benytter ikke dette, da encoding indtil videre kun benyttes til at estimere st�rrelsen p� det komprimerede billede, og filen alts� ikke gemmes p� disken.

