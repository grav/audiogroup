Programmet startes fra kommandolinien med en billedfil som parameter, f.eks. ./jpeg thomas.jpg

Med programmet følger to billeder, et udsnit af et fotografi i høj kvalitet, thomas.jpg, samt en stregtegning, stregtegning.png.

Når programmet startes, vises det indlæste billede to gange.

I programmets vindue er tre knapper samt to sliders. 

- Go to work
Knappen Go to work transformerer billedblokke om til frekvensdomænet vha. DCT, kvantiserer blokkene og transformerer blokkene tilbage til det spatiale domæne. Det kvantiserede billede vises nederst.

Efter endt transformation gives et estimat på komprimeringsgraden mellem den oprindelige, ukomprimerede billeddata og det kvantiserede, Huffman-encodede billede.

- Reset
Erstatter det kvantiserede billede med det oprindelige

- Diff
Fremhæver de steder på billedet, som er påvirket af kvantiseringen. Samtidigt beregnes PSNR.

- Matrix Size
Bestemmer blokstørrelsen ved transformering og kvantisering. Hvis størrelse 8 vælges, benyttes kvantiseringstabeller. Ved andre størrelser skæres koefficienter væk.

- Quality-slider
Denne slider bestemmer, hvor hårdt der skal kvantiseres. Ved brug af kvantiseringstabeller (blokstørrelse 8), definerer slideren det threshold som bestemmer hvorvidt koefficienter skal sættes til 0. Ellers bestemmer slideren størrelsen på det udsnit af koefficientmatricen som fjernes. Lavere Quality-værdi resulterer i hhv. mindre threshold og fjernelse af større udsnit.

