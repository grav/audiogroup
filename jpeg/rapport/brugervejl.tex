\section{Brugervejledning}
Programmets kildekode kan hentes via subversion vha. flg. kommando:

\texttt{svn http://svn.betafunk.dk/audiogroup/jpeg jpeg}

For at kompilere programmet skal man have installeret Qt, som gratis kan
hentes her:

\texttt{http://trolltech.no/downloads}

Programmet startes fra kommandolinien med en billedfil som parameter,
f.eks. \texttt{./jpeg thomas.jpg}

Med programmet f�lger to billeder, et udsnit af et fotografi i h�j kvalitet, \texttt{thomas.jpg}, samt en stregtegning, \texttt{stregtegning.png}.

N�r programmet startes, vises det indl�ste billede to gange.

I programmets vindue er tre knapper samt to sliders, som beskrives nedenfor:

\subsection*{Go to work}
Knappen Go to work transformerer billedblokke om til frekvensdom�net
vha. DCT, kvantiserer blokkene og transformerer blokkene tilbage til
det spatiale dom�ne. Det kvantiserede billede vises til h�jre.

Efter endt transformation gives et estimat p� komprimeringsgraden mellem den oprindelige, ukomprimerede billeddata og det kvantiserede, Huffman-encodede billede.

\subsection*{Reset}
Erstatter det kvantiserede billede med det oprindelige.

\subsection*{Diff}
Fremh�ver de steder p� billedet, som er p�virket af kvantiseringen. Samtidigt beregnes PSNR.

\subsection*{BlockSize}
Slideren bestemmer blokst�rrelsen ved transformering og kvantisering. Hvis st�rrelse 8 v�lges, benyttes kvantiseringstabeller. Ved andre st�rrelser sk�res koefficienter v�k.

\subsection*{Quality}
Denne slider bestemmer, hvor h�rdt der skal kvantiseres. Ved brug af kvantiseringstabeller (blokst�rrelse 8), definerer slideren det threshold som bestemmer hvorvidt koefficienter skal s�ttes til 0. Ellers bestemmer slideren st�rrelsen p� det udsnit af koefficientmatricen som fjernes. Lavere Quality-v�rdi resulterer i hhv. mindre threshold og fjernelse af st�rre udsnit.

