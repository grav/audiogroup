\section{DCT}
DCT er \textit{Discrete Cosine Transform} og er en anden m�de at lave frekvensanalyse p�, p� lignende vis som det kendes fra Fourier transformationen.\\
DCT findes i en lang r�kke forskellige implementationer, hver med deres styrker og svagheder.\\
Vi implementrede DCT-II og DCT-III som hhv. dct og idct.\\
DCT-II er defineret som:
\begin{displaymath}
X_k =
 \sum_{n=0}^{N-1} x_n \cos \left[\frac{\pi}{N} \left(n+\frac{1}{2}\right) k \right] \quad \quad k = 0, \dots, N-1.
\end{displaymath}

DCT-III er defineret som:
\begin{displaymath}
X_k = \frac{1}{2} x_0 +
 \sum_{n=1}^{N-1} x_n \cos \left[\frac{\pi}{N} n \left(k+\frac{1}{2}\right) \right] \quad \quad k = 0, \dots, N-1.
\end{displaymath}

DCT-II er den inverse til DCT-III og vice versa, hvilket g�r at vi kunne bruge dem til dct og idct.\\

Vores DCT implementationer kan ses i filen \texttt{dct.cc}
